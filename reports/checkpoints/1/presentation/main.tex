\documentclass[t]{beamer}  % [t], [c], или [b] --- вертикальное выравнивание на слайдах (верх, центр, низ)
%\documentclass[handout]{beamer} % Раздаточный материал (на слайдах всё сразу)
%\documentclass[aspectratio=169]{beamer} % Соотношение сторон

\usetheme{Rochester} % Тема оформления

\usecolortheme{default} % Цветовая схема

%%% Работа с русским языком
\usepackage{cmap}					% поиск в PDF
\usepackage{mathtext} 				% русские буквы в формулах
\usepackage[T2A]{fontenc}			% кодировка
\usepackage[utf8]{inputenc}			% кодировка исходного текста
\usepackage[english,russian]{babel}	% локализация и переносы

%% Beamer по-русски
\newtheorem{rtheorem}{Теорема}
\newtheorem{rproof}{Доказательство}
\newtheorem{rexample}{Пример}

%%% Дополнительная работа с математикой
\usepackage{amsmath,amsfonts,amssymb,amsthm,mathtools} % AMS
\usepackage{icomma} % "Умная" запятая: $0,2$ --- число, $0, 2$ --- перечисление

%% Номера формул
%\mathtoolsset{showonlyrefs=true} % Показывать номера только у тех формул, на которые есть \eqref{} в тексте.
%\usepackage{leqno} % Нумерация формул слева

%% Свои команды
\DeclareMathOperator{\sgn}{\mathop{sgn}}

%% Перенос знаков в формулах (по Львовскому)
\newcommand*{\hm}[1]{#1\nobreak\discretionary{}
{\hbox{$\mathsurround=0pt #1$}}{}}

%%% Работа с картинками
\usepackage{graphicx}  % Для вставки рисунков
\graphicspath{{images/}{images2/}}  % папки с картинками
\setlength\fboxsep{3pt} % Отступ рамки \fbox{} от рисунка
\setlength\fboxrule{1pt} % Толщина линий рамки \fbox{}
\usepackage{wrapfig} % Обтекание рисунков текстом

%%% Работа с таблицами
\usepackage{array,tabularx,tabulary,booktabs} % Дополнительная работа с таблицами
\usepackage{longtable}  % Длинные таблицы
\usepackage{multirow} % Слияние строк в таблице

%%% Программирование
\usepackage{etoolbox} % логические операторы

%%% Другие пакеты
\usepackage{lastpage} % Узнать, сколько всего страниц в документе.
\usepackage{soul} % Модификаторы начертания
\usepackage{csquotes} % Еще инструменты для ссылок

\usepackage[style=authoryear,maxcitenames=3,backend=biber,sorting=nty]{biblatex}
\addbibresource{bib.bib}
\AtBeginBibliography{\tiny} %  Сделаем библиографию меньшим шрифтом
\usepackage{multicol} % Несколько колонок

%%% Картинки
\usepackage{tikz} % Работа с графикой
\usepackage{pgfplots}
\usepackage{pgfplotstable}


\title{Исправление опечаток и грамматических ошибок в русскоязычных текстах при помощи BERT}
\author{Бунин Дмитрий, группа 792}
\date{Научный руководитель: Сорокин А. А.}

\begin{document}

\frame[plain]{\titlepage}	% Титульный слайд

\section{Анализ предметной области}
\subsection{Виды ошибок}
 
\begin{frame}
	\frametitle{\insertsection} 
	\framesubtitle{\insertsubsection}
	\begin{enumerate}
		\item Грамматические ошибки (grammatical)
		
		Нарушение правил грамматики. Например, неправильное образование и употребление форм слова.
		
		\item Орфографические ошибки (spelling)
		
		Неверное написание слов.
		
		\item Опечатки (typo)
		
		Ошибки в печатном тексте в результате случайности.
		
	\end{enumerate}
\end{frame}

\subsection{Задачи}
\begin{frame}
	\frametitle{\insertsection} 
	\framesubtitle{\insertsubsection}
	\begin{enumerate}
		\item GEC -- grammatical error correction
		\item Spelling correction
	\end{enumerate}
\end{frame}

\subsection{Метрики}
\begin{frame}
	\frametitle{\insertsection} 
	\framesubtitle{\insertsubsection}
	\begin{enumerate}
		\item F-мера
		
		Пусть имеется текст из $n$ предложений, тогда обозначим $g_i$ -- множество корректных исправлений предложения $i$, а $e_i$ -- множество наших исправлений. 
		\begin{gather*}
			R = \frac{\sum_{i=1}^n |g_i \cap e_i|}{\sum_{i=1}^n |g_i|},
			\\
			P = \frac{\sum_{i=1}^n |g_i \cap e_i|}{\sum_{i=1}^n |e_i|}.
		\end{gather*}
		\item GLEU
		
		Аналог BLEU для машинного перевода.
	\end{enumerate}
\end{frame}

\subsection{Датасеты}
\subsubsection{Английский язык}
\begin{frame}
	\frametitle{\insertsection} 
	\framesubtitle{\insertsubsection, \insertsubsubsection}
	\begin{enumerate}
		\item CoNLL-2014
			\begin{itemize}
				\item Статья: \textcite{Ng2013}
				\item Метрика: $F_{1/2}$
				\item Train: 1M токенов
				\item Test: 30k токенов
			\end{itemize}
		\item JFLEG
		\begin{itemize}
			\item Статья: \textcite{Napoles2017}
			\item Метрика: GLEU
			\item All: 1.5k предложений
		\end{itemize}
		\item BEA-2019
		\begin{itemize}
			\item \href{https://www.cl.cam.ac.uk/research/nl/bea2019st/}{Страница соревнования}
			\item Метрика: $F_{1/2}$
			\item Train: 628k токенов
			\item Validation: 87k токенов
			\item Test: 86k токенов
		\end{itemize}
	\end{enumerate}
\end{frame}

\subsubsection{Русский язык}
\begin{frame}
	\frametitle{\insertsection} 
	\framesubtitle{\insertsubsection,  \insertsubsubsection}
	\begin{enumerate}
		\item SpellRuEval
		\begin{itemize}
			\item Статья: \textcite{Sorokin2016a}
			\item \href{http://www.dialog-21.ru/en/evaluation/2016/spelling_correction/}{Страница соревнования}
			\item Validation: 2k предложений
			\item Test: 2k предложений
		\end{itemize}
	\end{enumerate}
\end{frame}

\subsection{Существующие подходы}
\begin{frame}[allowframebreaks]
	\frametitle{\insertsection} 
	\framesubtitle{\insertsubsection}
	\begin{enumerate}
		\item Стандартные решения
		\begin{itemize}
			\item GNU Aspell
			\item Hunspell
			\item JamSpell
			\item Яндекс.Спеллер
			
		\end{itemize}
		\item Модель шумного канала
			\begin{itemize}
				\item Модель на основе взвешенного расстояния Дамерау -- Левенштейна: \textcite{Kernighan1990}.
				\item Улучшенная модификация с более сложной моделью ошибок: \textcite{Brill2000}. 
			\end{itemize}
		
		\item Поиск кандидатов, ранжирование
		
		Схема была предложена в \textcite{Flor2012}. Решение задачи состоит из этапов:
		\begin{itemize}
			\item Поиск кандидатов для исправления ошибки
			\item Ранжирование кандидатов
		\end{itemize}
	
		\item Трансформеры
		
		Задачу можно рассматривать, как машинный перевод. Улучшения подобных моделей:
		\begin{itemize}
			\item Копирование исходного текста: \textcite{Zhao2019}.
			\item Генерация синтетических данных для обучения: \textcite{Kiyono2020}.
		\end{itemize}
	
		\item Sequence labeling
		
		При введении определенных классов трансформаций можно рассматривать GEC, как задачу sequence labeling. Модели:
		\begin{itemize}
			\item Parallel Iterative Edit Model: \textcite{Awasthi2020}.
			\item GECToR: \textcite{Omelianchuk2020}.
		\end{itemize}

	\end{enumerate}
\end{frame}

\subsection{Результаты}
\subsubsection{Английский язык}
\begin{frame}
	\frametitle{\insertsection} 
	\framesubtitle{\insertsubsection, \insertsubsubsection}
	\begin{table}
	\begin{center}
		\small
		\begin{tabular}{c c |c|c}
			\hline
			\textbf{GEC system} & \textbf{Ens.} & \textbf{CoNLL-2014 (test)}  & \textbf{BEA-2019 (test)}  \\
			\hline
			Copy Aug. Transformer &  & 59.8  & --  \\
			PIE &  & 59.7 & -- \\
			Transformer (synt. data) &  & 61.3 & 64.2  \\
			GECToR &  & \textbf{65.3} & \textbf{72.4}  \\
			\hline
			Copy Transformer & \checkmark & 61.2  & --  \\
			PIE & \checkmark & 61.2 & -- \\
			Transformer (synt. data) & \checkmark & 65.0 & 70.2  \\
			GECToR & \checkmark & \textbf{66.5} & \textbf{73.6}  \\
			\hline
		\end{tabular}
	\end{center}
	\caption{Сравнение результатов различных моделей для английского языка}
	\end{table}
\end{frame}

\subsubsection{Русский язык}
\begin{frame}
	\frametitle{\insertsection} 
	\framesubtitle{\insertsubsection, \insertsubsubsection}
		\begin{table}
		\begin{center}
			\small
			\begin{tabular}{c|c|c|c}
				\hline
				\textbf{GEC system} & \textbf{Precision}  & \textbf{Recall} & $\boldsymbol{F_1}$  \\
				\hline
				Yandex.Speller & \textbf{83.09}  & 59.86 & 69.59  \\
				JamSpell & 44.57 & 35.69 & 39.64 \\
				SpellRuEval Baseline & 55.91 & 46.41 & 50.72  \\
				SpellRuEval Winner & 81.98 & \textbf{69.25} & \textbf{75.07}  \\
				\hline
			\end{tabular}
		\end{center}
		\caption{Сравнение результатов различных моделей для русского языка}
	\end{table}
	
\end{frame}

\section{Исследование}
\subsection{Цель работы}
\begin{frame}
	\frametitle{\insertsection} 
	\framesubtitle{\insertsubsection}
	Исследование применимости модели BERT к задаче исправления опечаток и грамматических ошибок в русскоязычных текстах.
\end{frame}

\subsection{Архитектура модели}
\begin{frame}
	\frametitle{\insertsection} 
	\framesubtitle{\insertsubsection}
	
	Было решено начать с модели, основанной на ранжировании в связи с небольшим количеством данных и хорошими показателями согласно SpellRuEval (см.  \textcite{Sorokin2016}).
	
	Поиск кандидатов будет осуществляться на основании расстояния Дамерау -- Левенштейна при помощи префиксного бора. За основу будет взят \href{https://github.com/deepmipt/DeepPavlov/tree/0.12.1/deeppavlov/models/spelling_correction}{spelling-correction модуль} из библиотеки DeepPavlov.
	
	Полученные кандидаты будут отранжированны на основе признаков:
	\begin{enumerate}
		\item Взвешенное расстояние Дамерау -- Левенштейна (см. \textcite{Brill2000}).
		\item Вероятность BERT MLM (см. \textcite{Devlin2019}).
	\end{enumerate}

	Веса для BERT будут взяты из RuBERT.
	
\end{frame}

\subsection{План работ}
\begin{frame}
	\frametitle{\insertsection} 
	\framesubtitle{\insertsubsection}
	\begin{enumerate}
		\item Создание модели, ранжирующей на основе BERT MLM, ее тестирование на русскоязычных и англоязычных датасетах.
		\item Введение дополнительных признаков, как в решении-победителе SpellRuEval.
		\item Дообучение BERT на MLM для датасета.
		\item Возможно, испытание других языковых моделей (например, GPT).
	\end{enumerate}
\end{frame}

\section{Литература}
\begin{frame}[allowframebreaks]{reference}
	\frametitle{\insertsection} 
	\printbibliography
\end{frame}



\end{document}