\documentclass[a4paper,12pt]{report}

\usetheme{Rochester} % Тема оформления

\usecolortheme{default} % Цветовая схема

%%% Работа с русским языком
\usepackage{cmap}					% поиск в PDF
\usepackage{mathtext} 				% русские буквы в формулах
\usepackage[T2A]{fontenc}			% кодировка
\usepackage[utf8]{inputenc}			% кодировка исходного текста
\usepackage[english,russian]{babel}	% локализация и переносы

%% Beamer по-русски
\newtheorem{rtheorem}{Теорема}
\newtheorem{rproof}{Доказательство}
\newtheorem{rexample}{Пример}

%%% Дополнительная работа с математикой
\usepackage{amsmath,amsfonts,amssymb,amsthm,mathtools} % AMS
\usepackage{icomma} % "Умная" запятая: $0,2$ --- число, $0, 2$ --- перечисление

%% Номера формул
%\mathtoolsset{showonlyrefs=true} % Показывать номера только у тех формул, на которые есть \eqref{} в тексте.
%\usepackage{leqno} % Нумерация формул слева

%% Свои команды
\DeclareMathOperator{\sgn}{\mathop{sgn}}

%% Перенос знаков в формулах (по Львовскому)
\newcommand*{\hm}[1]{#1\nobreak\discretionary{}
{\hbox{$\mathsurround=0pt #1$}}{}}

%%% Работа с картинками
\usepackage{graphicx}  % Для вставки рисунков
\graphicspath{{images/}{images2/}}  % папки с картинками
\setlength\fboxsep{3pt} % Отступ рамки \fbox{} от рисунка
\setlength\fboxrule{1pt} % Толщина линий рамки \fbox{}
\usepackage{wrapfig} % Обтекание рисунков текстом

%%% Работа с таблицами
\usepackage{array,tabularx,tabulary,booktabs} % Дополнительная работа с таблицами
\usepackage{longtable}  % Длинные таблицы
\usepackage{multirow} % Слияние строк в таблице

%%% Программирование
\usepackage{etoolbox} % логические операторы

%%% Другие пакеты
\usepackage{lastpage} % Узнать, сколько всего страниц в документе.
\usepackage{soul} % Модификаторы начертания
\usepackage{csquotes} % Еще инструменты для ссылок

\usepackage[style=authoryear,maxcitenames=3,backend=biber,sorting=nty]{biblatex}
\addbibresource{bib.bib}
\AtBeginBibliography{\tiny} %  Сделаем библиографию меньшим шрифтом
\usepackage{multicol} % Несколько колонок

%%% Картинки
\usepackage{tikz} % Работа с графикой
\usepackage{pgfplots}
\usepackage{pgfplotstable}

%%% Нумерация изображений
\setbeamertemplate{caption}[numbered]


\title{Исправление опечаток и грамматических ошибок в русскоязычных текстах}
\author{Бунин Дмитрий Алексеевич}
\supervisor{Сорокин Алексей Андреевич}
\faculty{Физтех-школа Прикладной Математики и Информатики}
\department{Кафедра корпоративных информационных систем }

\begin{document}
	
	\titlepage
	
	\annotationpage{
		Исследование посвящено автоматическому исправлению опечаток и относится к области компьютерной лингвистики и автоматической обработки текста.  Целью работы является исследовать возможность применения модели BERT к исправлению опечаток в русскоязычных текстах на материале корпуса SpellRuEval. Выделим основные задачи исследования:
		\begin{enumerate}
			\item создание комплексной системы исправления опечаток с использованием BERT;
			\item изучение степени влияния вариантов модели BERT на улучшение качества решения задачи.
			\item изучение влияния других компонентов модели.
		\end{enumerate}
		
		Основным результатом работы является достигнутая величина F1-меры в 79.8\%, что превышает показатели как достигнутые в предыдущих работах, так и показываемые открытыми инструментами. Без использования BERT удалось добиться качества в 77.7\%, что без учета первой модели также является лучшим результатом на данный момент.
	}
	
	\tableofcontents
	
	% нумерация на каждой странице
	\pagestyle{plain}
	
	\import{chapters/}{1.introduction.tex}
	\import{chapters/}{2.task.tex}
	\import{chapters/}{3.overview.tex}
	\import{chapters/}{4.models_and_methods.tex}
	\import{chapters/}{5.experiments.tex}
	\import{chapters/}{6.conclusion.tex}
	
	% для цитирования при помощи пакета biblatex
	%\printbibliography[
	%	heading=bibintoc,
	%	title={Список литературы}
	%]
	
	% для цитирования при помощи пакета gost
	\newpage
	\addcontentsline{toc}{chapter}{Литература}
	\bibliography{bib}
	
\end{document}

