\chapter{Введение и обоснование актуальности}

Социальные медиа на данный момент являются крупнейшим источником новых текстов. В качестве примера можно рассмотреть комментарии на форумах, посты в соцсетях, подписи к фотографиям, публикации в блогах.
Многие тексты из этих источников содержат орфографические и пунктуационные ошибки, что может увеличивать вариативность используемого языка, осложняя автоматическую обработку. Так, согласно работе <<Моделирование расширенной лемматизации для русского языка на основе морфологического парсера TnT-Russian>>\cite{Shavrina2015} около 8\% несловарных слов в корпусе ГИКРЯ (Генеральный Интернет-Корпус Русского Языка, \url{http://www.webcorpora.ru/}) являются на самом деле результатом опечаток в словарных словах. Следует также отметить, что многие крупные корпуса, например ГИКРЯ, <<Тайга>> \cite{Shavrina2017}, содержат морфологический разбор предоставляемых предложений, при выполнении которого исправления ошибок не производилось. Это могло значительно повлиять на качество корпусной разметки.

Задача исправления опечаток и грамматических ошибок имеет также широкое применение в поисковых системах для исправления запроса. Так, в работе <<Автоматическое исправление опечаток в поисковых запросах без учета контекста>>\cite{Panina2013} было подсчитано, что около 12\% всех поисковых запросов к Яндексу содержат по крайней мере одну ошибку, а в работе <<Исправление поисковых запросов в Яндексе>> \cite{Bajtin2008} говорится о 15\%. Проверка орфографии также является неотъемлемой частью любого современного текстового редактора.

Решаемая задача имеет приложения и в сфере образования. Детектирование и исправление ошибок может помочь при оценке письменных ответов студентов. Например, в работе <<How to account for mispellings: Quantifying the benefit of character representations in neural content scoring models>> \cite{Riordan2019} было показано, что предварительное исправление опечаток улучшает качество оценивания.

Недавние исследования демонстрируют, что наличие опечаток может значительно повлиять на качество современных нейросетевых моделей. Авторами работы <<Noisy Text Data: Achilles’ Heel of BERT>> \cite{Kumar2020} было показано, что добавление в текст случайных опечаток негативно влияет на результаты модели BERT \cite{Devlin2019} при дообучении на задачах определения тональности (sentiment analysis) и схожести текстов (textual similarity).  

В свете вышеизложенных обстоятельств мы считаем достаточно важным изучать методы для исправления опечаток и грамматических ошибок в русскоязычных текстах.
