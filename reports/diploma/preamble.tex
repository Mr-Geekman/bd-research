%%% Работа с русским языком
\usepackage{cmap}					% поиск в PDF
\usepackage[T2A]{fontenc}			% кодировка
\usepackage[utf8]{inputenc}			% кодировка исходного текста
\usepackage[english,russian]{babel}	% локализация и переносы
\usepackage{indentfirst}
\frenchspacing

\renewcommand{\epsilon}{\ensuremath{\varepsilon}}
\renewcommand{\phi}{\ensuremath{\varphi}}
\renewcommand{\kappa}{\ensuremath{\varkappa}}
\renewcommand{\le}{\ensuremath{\leqslant}}
\renewcommand{\leq}{\ensuremath{\leqslant}}
\renewcommand{\ge}{\ensuremath{\geqslant}}
\renewcommand{\geq}{\ensuremath{\geqslant}}
\renewcommand{\emptyset}{\varnothing}

%%% Дополнительная работа с математикой
\usepackage{amsmath,amsfonts,amssymb,amsthm,mathtools} % AMS
\usepackage{icomma} % "Умная" запятая: $0,2$ --- число, $0, 2$ --- перечисление

%% Номера формул
%\mathtoolsset{showonlyrefs=true} % Показывать номера только у тех формул, на которые есть \eqref{} в тексте.
%\usepackage{leqno} % Нумереация формул слева

%% Свои команды
\DeclareMathOperator{\sgn}{\mathop{sgn}}

%% Перенос знаков в формулах (по Львовскому)
\newcommand*{\hm}[1]{#1\nobreak\discretionary{}
	{\hbox{$\mathsurround=0pt #1$}}{}}

%%% Работа с картинками
\usepackage{graphicx}  % Для вставки рисунков
\graphicspath{{images/}{images2/}}  % папки с картинками
\setlength\fboxsep{3pt} % Отступ рамки \fbox{} от рисунка
\setlength\fboxrule{1pt} % Толщина линий рамки \fbox{}
\usepackage{wrapfig} % Обтекание рисунков текстом

%%% Работа с таблицами
\usepackage{array,tabularx,tabulary,booktabs} % Дополнительная работа с таблицами
\usepackage{longtable}  % Длинные таблицы
\usepackage{multirow} % Слияние строк в таблице

%%% Теоремы
\theoremstyle{plain} % Это стиль по умолчанию, его можно не переопределять.
\newtheorem{theorem}{Теорема}[section]
\newtheorem{proposition}[theorem]{Утверждение}

\theoremstyle{definition} % "Определение"
\newtheorem{corollary}{Следствие}[theorem]
\newtheorem{problem}{Задача}[section]

\theoremstyle{remark} % "Примечание"
\newtheorem*{nonum}{Решение}

%%% Программирование
\usepackage{etoolbox} % логические операторы

%%% Страница
\usepackage{extsizes} % Возможность сделать 14-й шрифт
\usepackage{geometry} % Простой способ задавать поля
\geometry{top=20mm}
\geometry{bottom=20mm}
\geometry{left=30mm}
\geometry{right=15mm}

%\usepackage{fancyhdr} % Колонтитулы
% 	\pagestyle{fancy}
%\renewcommand{\headrulewidth}{0pt}  % Толщина линейки, отчеркивающей верхний колонтитул
% 	\lfoot{Нижний левый}
% 	\rfoot{Нижний правый}
% 	\rhead{Верхний правый}
% 	\chead{Верхний в центре}
% 	\lhead{Верхний левый}
%	\cfoot{Нижний в центре} % По умолчанию здесь номер страницы

\usepackage{setspace} % Интерлиньяж
\onehalfspacing % Интерлиньяж 1.5

% Отступ с красной строки
\setlength{\parindent}{1.25cm}

\usepackage{lastpage} % Узнать, сколько всего страниц в документе.

\usepackage{soul} % Модификаторы начертания

%%% Настройки ссылок
\usepackage{hyperref}
\usepackage[usenames,dvipsnames,svgnames,table,rgb]{xcolor}
\hypersetup{				% Гиперссылки
	unicode=true,           % русские буквы в раздела PDF
	pdftitle={Заголовок},   % Заголовок
	pdfauthor={Автор},      % Автор
	pdfsubject={Тема},      % Тема
	pdfcreator={Создатель}, % Создатель
	pdfproducer={Производитель}, % Производитель
	pdfkeywords={keyword1} {key2} {key3}, % Ключевые слова
	colorlinks=true,       	% false: ссылки в рамках; true: цветные ссылки
	linkcolor=red,          % внутренние ссылки
	citecolor=black,        % на библиографию
	filecolor=magenta,      % на файлы
	urlcolor=cyan           % на URL
}

%%% Цитирование
\usepackage{csquotes} % Еще инструменты для ссылок

% цитирование по-дефолту при помощи biblatex
%\usepackage[style=authoryear,maxcitenames=3,backend=biber,sorting=nty]{biblatex}

% цитирование при помощи biblatex для 
%\usepackage[maxcitenames=3,backend=biber,sorting=nty]{biblatex}
%\usepackage[style=gost-footnote, maxcitenames=3,backend=biber,sorting=nty]{biblatex}

% Цитирование при помощи пакета gost
\usepackage{cite}
\makeatletter % |список
\bibliographystyle{ugost2008} % |литературы
\renewcommand{\@biblabel}[1]{#1.}% |с
\makeatother % |точкой

%%% Работа с графикой
\usepackage{tikz} % Работа с графикой
\usepackage{pgfplots}
\usepackage{pgfplotstable}

\usepackage{import}

% Переименование TOC(table of conteints)
\addto\captionsrussian{% Replace "english" with the language you use
	\renewcommand{\contentsname}%
	{Содержание}%
}

%%% Оформление заголовков
% Главы
\usepackage{titlesec}    
\titleformat{\chapter}[display] % выбираем стандартный показ для /chapter
{\normalfont\filcenter\fontsize{14}{14}\bfseries} % настройки написания (shape)
{\chaptertitlename\ \thechapter} % формат написания
{10pt} % размер отступа между первой и второй строкой
{} 
{}

% Настройка отступов до/перед главой
\titlespacing*{\chapter}{0pt}{40pt}{20pt}

% Секции   
\titleformat{\section}[hang] % выбираем стандартный показ для /section
{\normalfont\filcenter\fontsize{14}{14}\bfseries} % настройки написания (shape)
{\thesection} % формат написания
{14pt} % размер отступа между номером и названием
{}

% Подсекции
\titleformat{\subsection}[hang] % выбираем стандартный показ для /section
{\normalfont\filcenter\fontsize{12}{12}\bfseries} % настройки написания (shape)
{\thesubsection} % формат написания
{12pt} % размер отступа между номером и названием
{}

%%% Оформление подписей к рисункам/таблицам
\usepackage{caption}
\captionsetup[figure]{labelsep=endash}
\captionsetup[table]{labelsep=endash, justification=raggedright, singlelinecheck=false}

%%% Нумерация на каждой странице%\usepackage{fancyhdr} % to change header and footers
%\usepackage{fancyhdr}
%\pagestyle{fancy} % Turn on the style
%\fancyhf{} % Start with clearing everything in the header and footer
%
%% Redefine plain style, which is used for chapter beginnings
%\fancypagestyle{plain}{%
%	\renewcommand{\headrulewidth}{0pt}%
%	\fancyhf{}%
%	\fancyfoot[C]{\thepage}%
%}
%
%% Redefine empty style, which is used for all other pages
%\fancypagestyle{empty}{%
%	\renewcommand{\headrulewidth}{0pt}%
%	\fancyhf{}%
%	\fancyfoot[C]{\thepage}%
%}

%%% Кастомный титульный лист
\renewcommand{\author}[1]{\def\mtauthortext{#1}}
\renewcommand{\title}[1]{\def\mttitletext{#1}}
\newcommand{\supervisor}[1]{\def\mtsupervisortext{#1}}
\newcommand{\faculty}[1]{\def\mtfacultytext{#1}}
\newcommand{\department}[1]{\def\mtdepartmenttext{#1}}

\renewcommand{\titlepage}{%
	\pagestyle{empty}
	
	\begin{center}
		{Федеральное государственное автономное образовательное учреждение высшего образования \\
			<<Московский физико-технический институт \\ (национальный исследовательский университет)>> \\
			\mtfacultytext \\
			\mtdepartmenttext}
	\end{center}
	
	\noindent
	\textbf{Направление подготовки}: 03.03.01\\
	\textbf{Направленность (профиль) подготовки}: Физика и компьютерные технологии
	
	\begin{center}
		\vspace{\fill}
		\singlespacing
		
		\Large \textbf{\mttitletext{}}\\
		
		\normalsize (бакалаврская работа)
		
		\vspace{\fill}
	\end{center}
	
	
	\begin{flushright}
		\begin{minipage}{0.4\linewidth}
			\textbf{Студент:} \\
			\mtauthortext 
			
			\rule{6.5cm}{1pt}\\
			\textit{(подпись студента)}
		\end{minipage}
	\end{flushright}
	
	\vspace{0em}
	
	\begin{flushright}
		\begin{minipage}{0.4\linewidth}
			\textbf{Научный руководитель:} \\
			\mtsupervisortext \\
			канд. физ.-мат. наук
			
			
			\rule{6.5cm}{1pt}\\
			\textit{(подпись научного руководителя)}
		\end{minipage}
	\end{flushright}
	
	
	\vspace{7em}
	
	\begin{center}
		Москва \\
		\the\year{}
	\end{center}
	
}

\newcommand{\annotationpage}[1]{
	\chapter*{Аннотация}
	#1
}
